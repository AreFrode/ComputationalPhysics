%%
%% Automatically generated file from DocOnce source
%% (https://github.com/hplgit/doconce/)
%%
%%


%-------------------- begin preamble ----------------------

\documentclass[%
oneside,                 % oneside: electronic viewing, twoside: printing
final,                   % draft: marks overfull hboxes, figures with paths
10pt]{article}

\listfiles               %  print all files needed to compile this document

\usepackage{relsize,makeidx,color,setspace,amsmath,amsfonts,amssymb}
\usepackage[table]{xcolor}
\usepackage{bm,ltablex,microtype}

\usepackage[pdftex]{graphicx}

\usepackage[T1]{fontenc}
%\usepackage[latin1]{inputenc}
\usepackage{ucs}
\usepackage[utf8x]{inputenc}

\usepackage{lmodern}         % Latin Modern fonts derived from Computer Modern

% Hyperlinks in PDF:
\definecolor{linkcolor}{rgb}{0,0,0.4}
\usepackage{hyperref}
\hypersetup{
    breaklinks=true,
    colorlinks=true,
    linkcolor=linkcolor,
    urlcolor=linkcolor,
    citecolor=black,
    filecolor=black,
    %filecolor=blue,
    pdfmenubar=true,
    pdftoolbar=true,
    bookmarksdepth=3   % Uncomment (and tweak) for PDF bookmarks with more levels than the TOC
    }
%\hyperbaseurl{}   % hyperlinks are relative to this root

\setcounter{tocdepth}{2}  % levels in table of contents

% --- fancyhdr package for fancy headers ---
\usepackage{fancyhdr}
\fancyhf{} % sets both header and footer to nothing
\renewcommand{\headrulewidth}{0pt}
\fancyfoot[LE,RO]{\thepage}
% Ensure copyright on titlepage (article style) and chapter pages (book style)
\fancypagestyle{plain}{
  \fancyhf{}
  \fancyfoot[C]{{\footnotesize \copyright\ 1999-2019, "Computational Physics I FYS3150/FYS4150":"http://www.uio.no/studier/emner/matnat/fys/FYS3150/index-eng.html". Released under CC Attribution-NonCommercial 4.0 license}}
%  \renewcommand{\footrulewidth}{0mm}
  \renewcommand{\headrulewidth}{0mm}
}
% Ensure copyright on titlepages with \thispagestyle{empty}
\fancypagestyle{empty}{
  \fancyhf{}
  \fancyfoot[C]{{\footnotesize \copyright\ 1999-2019, "Computational Physics I FYS3150/FYS4150":"http://www.uio.no/studier/emner/matnat/fys/FYS3150/index-eng.html". Released under CC Attribution-NonCommercial 4.0 license}}
  \renewcommand{\footrulewidth}{0mm}
  \renewcommand{\headrulewidth}{0mm}
}

\pagestyle{fancy}


% prevent orhpans and widows
\clubpenalty = 10000
\widowpenalty = 10000

% --- end of standard preamble for documents ---


% insert custom LaTeX commands...

\raggedbottom
\makeindex
\usepackage[totoc]{idxlayout}   % for index in the toc
\usepackage[nottoc]{tocbibind}  % for references/bibliography in the toc

%-------------------- end preamble ----------------------

\begin{document}

% matching end for #ifdef PREAMBLE

\newcommand{\exercisesection}[1]{\subsection*{#1}}


% ------------------- main content ----------------------



% ----------------- title -------------------------

\thispagestyle{empty}

\begin{center}
{\LARGE\bf
\begin{spacing}{1.25}
Project 5, deadline  December 15, 2019
\end{spacing}
}
\end{center}

% ----------------- author(s) -------------------------

\begin{center}
{\bf \href{{http://www.uio.no/studier/emner/matnat/fys/FYS3150/index-eng.html}}{Computational Physics I FYS3150/FYS4150}}
\end{center}

    \begin{center}
% List of all institutions:
\centerline{{\small Department of Physics, University of Oslo, Norway}}
\end{center}
    
% ----------------- end author(s) -------------------------

% --- begin date ---
\begin{center}
Fall semester 2019
\end{center}
% --- end date ---

\vspace{1cm}


\subsection*{Studies of social interactions using the Ising model}

The Ising model is one of the most
frequently used models of statistical mechanics. Recently, this
model has also become widely studied in many other disciplines, including mathematics, statistics, biology, \href{{https://www.springer.com/gp/book/9783319477046}}{economy} and \href{{https://www.springer.com/gp/book/9781461420316}}{sociology}.

\textbf{For this project you can hand in collaborative reports and programs.}
This project (together with projects 3 and 4) counts 1/3 of the final mark.



\paragraph{Introduction.}
The aim of this project is to use the Ising model from project 4 and 
apply it to the modeling of electoral patterns and social interactions.
This project aims at studying the work of \href{{https://arxiv.org/abs/cond-mat/0101130}}{Katarzyna Sznajd-Weron and
Jozef Sznajd}, see also their
published work in \href{{https://www.worldscientific.com/doi/abs/10.1142/S0129183100000936}}{Int. J. Mod. Phys. C11,(2000) 1157}. 


We repeat 
In its simplest form
the energy of the Ising model from project 4 without an externally applied magnetic field, 
\[
E=-J\sum_{< kl >}^{N}s_ks_l 
\]
with
$s_k=\pm 1$. The quantity $N$ represents the total number of spins and $J$ is a coupling
constant expressing the strength of the interaction between
neighboring spins.  The symbol $<kl>$ indicates that we sum over
nearest neighbors only. 
ordering, viz $J> 0$.  

We will  consider as our model a community which time and again should
take a stand in some matter, for example vote on a president in
a two-party system. If each member of the
community can take only two attitudes ($A$ or $B$) then in several
votes one expects some difference $m$ of voters for $A$ and against.
We assume three limiting cases:

\begin{enumerate}
\item all members of the community vote for $A$ (an \textbf{all} $A$ state),

\item all members of the community vote for $B$ (an \textbf{all} $B$ state),

\item $50\%$ vote for $A$ and $50\%$ vote for $B$,
\end{enumerate}

\noindent
where the latter is meant to  be the stable solutions of our model.

The aim here is to analyze the time evolution of $m$. To model the
above mentioned system we consider an Ising spins chain
($S_i;i=1,2,\ldots N$) with the following dynamic rules:

\begin{itemize}
\item if $S_iS_{i+1}=1$ then $S_{i-1}$ and $S_{i+2}$ take the direction of the pair (i,i+1),

\item if $S_iS_{i+1}=-1$ then $S_{i-1}$ takes the direction of $S_{i+1}$ and $S_{i+2}$ the direction of $S_i$. 
\end{itemize}

\noindent
These rules describe the influence of a given pair on the decision of
its nearest neighbours. When members of a pair have the same opinion
then their nearest neighbours agree with them. On the contrary, when
members of a pair have opinions different then the nearest neighbour
of each member disagrees with her/him.  These dynamic rules lead to
the three steady states above.  However, the third steady state ($50\%$
for $A$ and $50\%$ for $B$) is realized in a very special way.  Every
member of the community disagrees with her/his nearest neighbour (it
is easy to see that the Ising model with only next nearest neighbour
interaction has such fixed points: ferro- and antiferromagnetic
state).  According to the authors of the above mentioned article,
this rule is in accordance with the well known sentence
\emph{united we stand divided we fall} (USDF-model).


\paragraph{Project 5a):}
To investigate our model we perform a standard Monte Carlo simulation
with random updating.  Consider a chain of $N$ Ising spins with free
boundary conditions. In our simulation you could use values from
$N=1000$ up to $N=10000$. We start from a totally random initial state
i.e.~to each site of the chain we assign an arrow with a randomly
chosen direction: up or down (Ising spin).  Show that you obtain as a
state, one of the three fixed points (1-3, i.e.~AAAA, BBBB, ABAB) with
probability 0.25,0.25 and 0.5, respectively. The typical relaxation
time for $N=1000$ is $\sim 10^4$ Monte Carlo steps (MCS). Plot the
spatial distribution of spins from the initial to a steady state and
compare your results with figure 1 of \href{{https://arxiv.org/abs/cond-mat/0101130}}{Katarzyna Sznajd-Weron and
Jozef Sznajd}. Can you see
whether there is a formation of clusters?  Comment your results.


\paragraph{Project 5b): Magnetization and Social Mood.}
Let us define the decision as a magnetization, i.e.:
\begin{equation}
m=\frac{1}{N}\sum_{i=1}^N S_i.
\label{em}
\end{equation}

Compute the magnetization and compare your results with figure 2 of \href{{https://arxiv.org/abs/cond-mat/0101130}}{Katarzyna Sznajd-Weron and
Jozef Sznajd}.

Without any external stimulation decision can change
dramatically in a relatively short time.
Such strongly non-monotonic behaviour of the change of $m$
is typically observed in the USDF-model when the system evolves
towards the third steady state (total disagreement or in magnetic
language the antiferromagnetic state). Comment your results.


\paragraph{Project 5c): Autocorrelation Function.}
To measure the time correlation of $m$ one can employ the
classical autocorrelation function:

\begin{equation}
G(\Delta t) = \frac
{\sum \left( m(t)-<m>\right) \left( m(t+ \Delta t)- <m>\right)}
{\sum (m(t)-<m>)^2}.
\end{equation}

In the work of Sznajd-Weron and Snajzd, there is a comparison of
simulation results with empirical data as shown in their figure 3. Make a plot similar to their figure 3 and comment your results. 
Following the changes of one particular individual, the dynamics seem to lead to some interesting effects. If an individual changes her/his opinion at time $t$ she/he will probably
change it also at time $t+1$.

 On the other hand
an individual can stay for a long time without changing her/his decision.
Let us denote by $\tau$ the time needed by an individual to change  her/his
opinion. From the autocorrelation function  it can be seen that $\tau$ is usually very short,
but sometimes can be very long. The distribution of $(\tau)$ ($P(\tau)$)
follows seeminingly a power law behavior
with an exponent $-3/2$. Plot this distribution and compare your results with
figure 4 Sznajd-Weron and Snajzd. Comment your results.





\paragraph{Project 5d): Initial Conditions.}
We will now study the influence of the initial conditions on the
evolution of the system by considering  two different ways -
randomly and in clusters.  In both cases you could start from an initial
concentration $c_B$ of opinion $B$.  In the random setup $c_B*N$
individuals are randomly (uniformly) chosen out of all $N$
individuals. In the cluster setup simply the first $c_B*N$ individuals
are chosen.

Study whether the distribution of decision time $\tau$ still follows
the power law with the same exponent as you found in the previous
part.  A non-monotonic behaviour of decision change is still typical
and sometimes even much stronger.  However, it is obvious that if
initially there is more $A$'s then $B$'s the final state should be
more often "all $A$" then "all $B$".  Compare your results to figure 5
of Sznajd-Weron and Snajzd and comment your results.

\paragraph{Project 5e): Information Noise.}
It is well known that the changes of opinion are determined by the
social impact. Till now we have considered a community in which a
change of an individuals opinion is caused only by a contact with its
neighbours. It was the simplest social impact one can imagine.  Now,
we introduce to our model noise $p$, which is the probability that an
individual, instead of following the dynamic rules, will make a random
decision. We start from an "all $A$" state to investigate if there is
a $p \in (0,1)$ which does not throw the system out of this state.

See if you can reproduce figures 6 and 7 of the above authors and
comment your final results. Their discussion section contains several interesting observations you may consider discussing. The text by \href{{https://www.springer.com/gp/book/9781461420316}}{Serge Galam} contains several other interesting obeservations which can be useful in the present analysi.

\subsection*{Introduction to numerical projects}

Here follows a brief recipe and recommendation on how to write a report for each
project.

\begin{itemize}
  \item Give a short description of the nature of the problem and the eventual  numerical methods you have used.

  \item Describe the algorithm you have used and/or developed. Here you may find it convenient to use pseudocoding. In many cases you can describe the algorithm in the program itself.

  \item Include the source code of your program. Comment your program properly.

  \item If possible, try to find analytic solutions, or known limits in order to test your program when developing the code.

  \item Include your results either in figure form or in a table. Remember to        label your results. All tables and figures should have relevant captions        and labels on the axes.

  \item Try to evaluate the reliabilty and numerical stability/precision of your results. If possible, include a qualitative and/or quantitative discussion of the numerical stability, eventual loss of precision etc.

  \item Try to give an interpretation of you results in your answers to  the problems.

  \item Critique: if possible include your comments and reflections about the  exercise, whether you felt you learnt something, ideas for improvements and  other thoughts you've made when solving the exercise. We wish to keep this course at the interactive level and your comments can help us improve it.

  \item Try to establish a practice where you log your work at the  computerlab. You may find such a logbook very handy at later stages in your work, especially when you don't properly remember  what a previous test version  of your program did. Here you could also record  the time spent on solving the exercise, various algorithms you may have tested or other topics which you feel worthy of mentioning.
\end{itemize}

\noindent
\subsection*{Format for electronic delivery of report and programs}

The preferred format for the report is a PDF file. You can also use DOC or postscript formats or as an ipython notebook file.  As programming language we prefer that you choose between C/C++, Fortran2008 or Python. The following prescription should be followed when preparing the report:

\begin{itemize}
  \item Use Devilry to hand in your projects, log in  at  \href{{http://devilry.ifi.uio.no}}{\nolinkurl{http://devilry.ifi.uio.no}} with your normal UiO username and password and choose either 'fys3150' or 'fys4150'. There you can load up the files within the deadline.

  \item Upload \textbf{only} the report file!  For the source code file(s) you have developed please provide us with your link to your github domain.  The report file should include all of your discussions and a list of the codes you have developed.  Do not include library files which are available at the course homepage, unless you have made specific changes to them.

  \item In your git repository, please include a folder which contains selected results. These can be in the form of output from your code for a selected set of runs and input parametxers.

  \item In this and all later projects, you should include tests (for example unit tests) of your code(s).

  \item Comments  from us on your projects, approval or not, corrections to be made  etc can be found under your Devilry domain and are only visible to you and the teachers of the course.
\end{itemize}

\noindent
Finally, 
we encourage you to work two and two together. Optimal working groups consist of 
2-3 students. 








% ------------------- end of main content ---------------

\end{document}

